\documentclass[12pt]{article}

\usepackage{fouriernc}
\usepackage{amssymb}
\usepackage{amsmath}
\usepackage{amsfonts}
\usepackage[utf8]{inputenc}
\usepackage[T1]{fontenc}
\usepackage[margin=1in]{geometry}
\usepackage{graphicx}

\graphicspath{ {./images/} }

\newcommand{\curly}[1]{\left\{      #1 \right\}     }
\newcommand{\round}[1]{\left(       #1 \right)      }
\newcommand{\hard} [1]{\left[       #1 \right]      }
\newcommand{\abs}  [1]{\left|       #1 \right|      }
\newcommand{\floor}[1]{\left\lfloor #1 \right\rfloor}
\newcommand{\ceil} [1]{\left\lceil  #1 \right\rceil }
\newcommand{\R}    [0]{\mathbb{R}                   }
\newcommand{\Z}    [0]{\mathbb{Z}                   }
\newcommand{\N}    [0]{\mathbb{N}                   }

\setlength{\parindent}{0in}

\title{Homework 4}
\author{Tim Harding}

\begin{document}
\maketitle

\section*{3.2.8}

\subsection*{Problem}

Show that the sequence $\frac{1}{2}$, $-\frac{1}{5}$, $\frac{1}{8}$, $-\frac{1}{11}$, $\frac{1}{14}$ converges to 0.

\subsection*{Solution}
With a starting index of 0, we can write $a_n = \frac{(-1)^n}{2 + 3n}$. Then we show that $\lim_{n\to\infty} a_n = 0$:

$\forall \epsilon \in \R : \epsilon > 0$, $\exists N = \ceil{\frac{1}{3\epsilon} - \frac{2}{3}}$, $\forall n \in \N : n > N \implies \abs{\frac{(-1)^n}{2+3n} - 0} = \frac{1}{2+3n} < \epsilon$.



\section*{3.2.9}

\subsection*{Problem}
If $\frac{\cos(n)}{n}$ converges, prove its limit.

\subsection*{Solution}
We show that $\lim_{n\to\infty} \frac{\cos(n)}{n} = 0$:

$\forall \epsilon \in \R : \epsilon > 0$, $\exists N = \ceil{\frac{1}{\epsilon}}$, $\forall n \in \N : n > N \implies \abs{\frac{\cos(n)}{n} - 0} \leq \abs{\frac{1}{n}} = \frac{1}{n} < \epsilon$.




\section*{3.2.10}

\subsection*{Problem}
Investigate the convergence of $\floor{1 - \frac{1}{n}}$.

\subsection*{Solution}
The difference will always be less that 1 but greater than or equal to zero whenever $n \geq 1$, so every element of the sequence is just 0 and the sequence converges to 0.



\section*{3.2.11}

\subsection*{Problem}
Show that $\forall a \in \R : \abs{a} < 1 \implies \lim_{n \to \infty} a^n = 0$. Discuss the case of $\abs{a} = 1$.

\subsection*{Solution}




\section*{3.2.12}

\subsection*{Problem}
Prove that $\lim_{n \to \infty} \frac{n^2 + n}{2n^2 + 1} = \frac{1}{2}$.

\subsection*{Solution}




\section*{3.2.15}

\subsection*{Problem}
Show that if a sequence is convergent then its limit is unique.

\subsection*{Solution}




\section*{3.2.A}

\subsection*{Problem}
Prove that $\frac{1}{n}$ is Cauchy. Specifically, when we need to get $|1/m - 1/n| < \epsilon$, try to estimate as $|1/m - 1/n| < 1/m$ and show why this would work as long as we assume $m<n$ without loss of generality (since if $n<m$, we can use $|1/m - 1/n| < 1/n$).

\subsection*{Solution}



\section*{3.3.14}

\subsection*{Problem}
Use quantifiers to negate the following definitions:

\begin{enumerate}
    \item We say that the sequence $a_n$ diverges to infinity and we write $\lim_{n \to \infty} a_n = \infty$ if for every real $B$ there exists a number $N$ such that for all $n \in \N$ with $n > N$, $a_n > B$ is true.
    \item We say that the sequence $a_n$ diverges to negative infinity and we write $\lim_{n \to \infty} a_n = -\infty$ if for every real $B$ there exists a number $N$ such that for all $n \in \N$ with $n > N$, $a_n < B$ is true.
\end{enumerate}

\subsection*{Solution}

\subsubsection*{1}
We negate the statement as such:
\begin{align*}
    \neg (\forall B \in \R,\ \exists N \in \N^+,\ \forall n \in \N^+ : n > N,\ a_n > B &\implies \lim_{n\to\infty} a_n = \infty) \\
    = \exists B \in \R,\ \forall N \in \N^+,\ \exists n \in \N^+ : n > N,\ a_n \leq B &\implies \lim_{n\to\infty} a_n \neq \infty
\end{align*}

\subsubsection*{2}
We negate the statement as such:
\begin{align*}
    \neg (\forall B \in \R,\ \exists N \in \N^+,\ \forall n \in \N^+ : n > N,\ a_n < B &\implies \lim_{n\to\infty} a_n = -\infty) \\
    = \exists B \in \R,\ \forall N \in \N^+,\ \exists n \in \N^+ : n > N,\ a_n \geq B &\implies \lim_{n\to\infty} a_n \neq -\infty
\end{align*}



\section*{3.3.16}

\subsection*{Problem}
Find the $n$th term of the sequence 1, 1, 2, 2, 3, 3, 4, 4, 5, 5, $\cdots$ and show that it diverges to infinity.

\subsection*{Solution}
With a starting index of 1, we have $a_n = \ceil{\frac{n}{2}}$. We then continue with the proof:
$\forall M \in \R : M > 0$, $\exists N = 2\ceil{M}$, $\forall n \in \N^+ : n > N \implies \ceil{\frac{n}{2}} > M$.



\section*{3.3.17}

\subsection*{Problem}
Show that the sequence $a_n = \sqrt{n} - n$ diverges to negative infinity.

\subsection*{Solution}
$\forall M \in \R : M < 0$, $\exists N = \max\round{\ceil{(1-M)^2}, 2}$, $\forall n \in \N^+ : n > N \implies \sqrt{n} - n = \sqrt{n} \round{1 - \sqrt{n}} < 1 - \sqrt{n} < M$.




\section*{3.3.A}

\subsection*{Problem}
Prove that $\lim_{n \to \infty} -\ln(n) = -\infty$.

\subsection*{Solution}
$\forall M \in \R : M < 0$, $\exists N = \ceil{e^{-M}}$, $\forall n \in \N^+ : n > N \implies -\ln(n) < M$.



\section*{3.3.B}

\subsection*{Problem}
Prove that $\lim_{n \to \infty} (-1)^{n} n$ does not diverge to either $\infty$ or $-\infty$.

\subsection*{Solution}



\section*{3.4.21}

\subsection*{Problem}
Prove the following:
\begin{enumerate}
    \item \begin{align*}
        \round{\abs{a_n} \leq b_n} \wedge \round{\lim_{n \to \infty} b_n = 0} \implies \lim_{n \to \infty} a_n = 0
    \end{align*}
    \item \begin{align*}
        \lim_{n \to \infty} \abs{a_n} = 0 \implies \lim_{n \to \infty} a_n = 0
    \end{align*}
\end{enumerate}

\subsection*{Solution}

\subsubsection*{1}
We have that $-b_n \leq a_n \leq b_n$ and $b_n \to 0$, so from the squeeze theorem we have that $a_n \to 0$.

\subsubsection*{2}
In the limit, we have that $-0 \leq a_n \leq 0$, so from the squeeze theorem we have that $a_n \to 0$.



\section*{3.4.22}

\subsection*{Problem}
Show that
\begin{align*}
    \lim_{n \to \infty} \frac{(-1)^n \ln(n)}{\sqrt{n}} = 0
\end{align*}

\subsection*{Solution}
We know that
\begin{align*}
    -\frac{\ln(n)}{\sqrt{n}} \leq \frac{(-1)^n \ln(n)}{\sqrt{n}} \leq \frac{\ln(n)}{\sqrt{n}}
\end{align*}
Using l'Hopital's rule, we evaluate
\begin{align*}
     & \lim_{n\to\infty} \frac{\ln(n)}{\sqrt{n}} \\
    =& \lim_{n\to\infty} \frac{\frac{1}{n}}{\frac{1}{2\sqrt{n}}} \\
    =& \lim_{n\to\infty} \frac{2\sqrt{n}}{n} \\
    =& 2 \lim_{n\to\infty} n^{-\frac{1}{2}} \\
    =& 0
\end{align*}
By using the linearity property of limits, we also have $\lim_{n\to\infty} -\frac{\ln(n)}{\sqrt{n}} = 0$. Combining the preceeding facts using the squeeze theorem, we have that $\lim_{n\to\infty} \frac{(-1)^n \ln(n)}{\sqrt{n}} = 0$.



\section*{3.4.23}

\subsection*{Problem}
Find the limit or divergence behavior of each of the following:
\begin{enumerate}
    \item $\frac{n - 2}{2n + 7}$
    \item $\frac{(-1)^n (n - 2)}{2n + 7}$
    \item $\frac{n - 2}{2n^2}$
    \item $\frac{n^2}{e^n}$
    \item $n^2 \round{\frac{1}{2}}^n$
    \item $\sin(\pi n + \frac{\pi}{2})$
    \item $n^\frac{1}{n}$
    \item $\round{1 - \frac{2}{n}}^n$
    \item $\sqrt{n^2 + 1} - n$
    \item $n \sin\round{\frac{1}{n}}$
\end{enumerate}

\subsection*{Solution}

\subsubsection*{1}
\begin{align*}
     & \lim_{n\to\infty} \frac{n - 2}{2n + 7} \\
    =& \lim_{n\to\infty} \frac{1}{2} \\
    =& \frac{1}{2}
\end{align*}

\subsubsection*{2}
In the limit, this function oscillates between $-\frac{1}{2}$ and $\frac{1}{2}$ and does not converge.

\subsubsection*{3}
\begin{align*}
     & \lim_{n\to\infty} \frac{n-2}{2n^2} \\
    =& \lim_{n\to\infty} \frac{1}{4n} \\
    =& 0
\end{align*}

\subsubsection*{4}
\begin{align*}
     & \lim_{n\to\infty} \frac{n^2}{e^n} \\
    =& \lim_{n\to\infty} \frac{2n}{e^n} \\
    =& \lim_{n\to\infty} \frac{2}{e^n} \\
    =& 0
\end{align*}

\subsubsection*{5}
\begin{align*}
     & \lim_{n\to\infty} n^2 \round{\frac{1}{2}}^n \\
    =& \lim_{n\to\infty} \frac{n^2}{2^n} \\
    =& \lim_{n\to\infty} \frac{2n}{2^n \ln(2)} \\
    =& \frac{2}{\ln(2)} \lim_{n\to\infty} \frac{n}{2^n} \\
    =& \frac{2}{\ln(2)} \lim_{n\to\infty} \frac{1}{2^n \ln(2)} \\
    =& \frac{2}{\ln^2(2)} \lim_{n\to\infty} 2^{-n} \\
    =& 0
\end{align*}

\subsubsection*{6}
We have
\begin{align*}
    \sin\round{\pi n + \frac{\pi}{2}} = (-1)^n
\end{align*}
Which does not converge but rather oscillates between -1 and 1.

\subsubsection*{7}
\begin{align*}
     & \lim_{n\to\infty} n^\frac{1}{n} \\
    =& \lim_{n\to\infty} \round{e^{\ln(n)}}^\frac{1}{n} \\
    =& \lim_{n\to\infty} e^{\frac{1}{n} \ln(n)}
\end{align*}
We then evaluate
\begin{align*}
     & \lim_{n\to\infty} \frac{\ln(n)}{n} \\
    =& \lim_{n\to\infty} \frac{\frac{1}{n}}{1} \\
    =& 0
\end{align*}
and substitute back
\begin{align*}
     & \lim_{n\to\infty} e^{\frac{1}{n} \ln(n)} \\
    =& e^0 \\
    =& 1
\end{align*}

\subsubsection*{8}
\begin{align*}
     & \lim_{n\to\infty} \round{1 - \frac{2}{n}}^n \\
    =& \lim_{n\to\infty} \round{\frac{n-2}{n}}^n \\
    =& \round{\lim_{n\to\infty} \frac{n-2}{n}}^n \\
    =& \round{\lim_{n\to\infty} \frac{1}{1}}^n \\
    =& 1^n \\
    =& 1
\end{align*}

\subsubsection*{9}
\begin{align*}
     & \lim_{n\to\infty} \round{\sqrt{n^2 + 1} - n} \\
    =& \lim_{n\to\infty} \round{\sqrt{n^2 + 1} - n}\frac{\sqrt{n^2 + 1} + n}{\sqrt{n^2 + 1} + n} \\
    =& \lim_{n\to\infty} \frac{n^2 + 1 - n^2}{\sqrt{n^2 + 1} + n} \\
    =& \lim_{n\to\infty} \frac{1}{\sqrt{n^2 + 1} + n} \\
    =& 0
\end{align*}

\subsubsection*{10}
\begin{align*}
     & \lim_{n\to\infty} n \sin\round{\frac{1}{n}} \\
    =& \lim_{n\to\infty} \frac{\sin\round{n^{-1}}}{n^{-1}} \\
    =& \lim_{n\to\infty} \frac{\cos\round{n^{-1}} \times -n^{-2}}{-n^{-2}} \\
    =& \lim_{n\to\infty} \cos\round{n^{-1}} \\
    =& \cos(0) \\
    =& 1
\end{align*}



\section*{3.4.25}

\subsection*{Problem}
Investigate the convergence of
\begin{enumerate}
    \item $\round{\frac{1}{2}}^n$
    \item $\frac{27^n}{(n!)^3}$
    \item $n \round{\frac{1}{2}}^n$
    \item $\frac{n - 2}{n + 2}$
    \item $\frac{a^n}{n!}$
\end{enumerate}
using the fact that a bounded and monotonic sequence is convergent. Find the limits where possible.

\subsection*{Solution}

\subsubsection*{1}
We know that the limit exists because $0 \leq \frac{1}{2^n} \leq 1$ for $n \geq 1$ and the sequence is decreasing monotonically by half at each step.
\begin{align*}
     & \lim_{n\to\infty} \round{\frac{1}{2}}^n \\
    =& \lim_{n\to\infty} 2^{-n} \\
    =& 0
\end{align*}

\subsubsection*{2}
We know that the limit exists because $0 \leq \frac{27^n}{(n!)^3} \leq 27$ for $n \geq 1$ and the denominator is growing faster than the numerator for $n > 3$.
\begin{align*}
     & \lim_{n\to\infty} \frac{27^n}{(n!)^3} \\
    =& \lim_{n\to\infty} \round{\frac{3^n}{n!}}^3 \\
    =& \round{\lim_{n\to\infty} \frac{3^n}{n!}}^3 \\
    =& 0
\end{align*}

\subsubsection*{3}
We know that the limit exists because the denominator is growing faster than the numerator so we have a monotonic decreasing function and because $0 \leq \frac{n}{2^n} \leq \frac{1}{2}$.
\begin{align*}
     & \lim_{n\to\infty} n \round{\frac{1}{2}}^n \\
    =& \lim_{n\to\infty} \frac{n}{2^n} \\
    =& \lim_{n\to\infty} \frac{1}{2^n \ln(2)} \\
    =& \frac{1}{\ln(2)} \lim_{n\to\infty} 2^{-n} \\
    =& 0
\end{align*}

\subsubsection*{4}
We know that $-\frac{1}{3} \leq \frac{n-2}{n+2} \leq 1$ and the ratio is increasing monotonically because the numerator, being a smaller number, moves a larger percentage from 0 than does the denominator at each step. Therefore, the limit exists.
\begin{align*}
     & \lim_{n\to\infty} \frac{n - 2}{n + 2} \\
    =& \lim_{n\to\infty} \frac{1}{1} \\
    =& 1
\end{align*}

\subsubsection*{5}
The function is monotonic decreasing for $n \geq \ceil{a}$ and is bounded by $0 \leq \frac{a^n}{n!} \leq \frac{a^a}{a!}$ so we know that the limit exists. We know that in the limit, $n!$ grows faster than $a^n$, so
\begin{align*}
    \lim_{n\to\infty} \frac{a^n}{n!} = 0
\end{align*}



\section*{3.4.29}

\subsection*{Problem}
Show that $\forall n \in \N$, $a_n \leq b_n$, $\lim_{n \to \infty} a_n = L$, $\lim_{n \to \infty} b_n = M \implies L \leq M$. Use this result to prove that the limit is unique for a convergent sequence.

\subsection*{Solution}
$\forall \epsilon \in \R : \epsilon > 0$, $\forall n \in \N \implies a_n - b_n \leq 0$, $\exists K_1, K_2 \in \N$, $\forall n \in \N : n > K_1 \implies L - a_n < \epsilon$, $\forall n \in \N : n > K_2 \implies M - b_n < \epsilon$, $\exists K = \max(K_1, K_2)$, $\forall n \in \N : n > K \implies L - M \leq L - M - (a_n - b_n) = (L - a_n) - (M - b_n)$.

I'm having trouble finishing this prove with an epsilon argument but it makes sense that if every term of $a_n$ is less than that of $b_n$, $a_n$ certainly cannot exceed $b_n$ after infinitely many terms. $L$ may equal $M$ if, for example, $a_n = n^{-3}$ and $b_n = n^{-2}$, or $L$ can be less than $M$ for other sequences such as $a_n = n^{-2}$ and $b_n = 2 n^{-2}$.



\section*{3.4.31}

\subsection*{Problem}
Use the fact that if $\lim_{n \to \infty} a_n = L$ and $f : \R \to \R$ is continuous at $L$ then $\lim_{n \to \infty} f(a_n) = f(L)$ to prove that $\sqrt{1 - \frac{1}{n}}$ converges to 1.

\subsection*{Solution}
We let $f(x) = \sqrt{1-x}$ and $a_n = \frac{1}{n}$. Note that $\lim_{n\to\infty} \frac{1}{n} = 0$ so $L = 0$.
\begin{align*}
    \lim_{n\to\infty} \sqrt{1 - \frac{1}{n}} = \lim_{n\to\infty} f(a_n) = f(0) = \sqrt{1 - 0} = 1
\end{align*}



\section*{3.4.32}

\subsection*{Problem}
Use the fact that if $\lim_{n \to \infty} a_n = L$ and $f : \R \to \R$ is continuous at $L$ then $\lim_{n \to \infty} f(a_n) = f(L)$ to find the limit of $2^\frac{1}{n}$.

\subsection*{Solution}
We let $f(x) = 2^x$ and $a_n = \frac{1}{n}$. Note that $\lim_{n\to\infty} \frac{1}{n} = 0$ so $L = 0$. Then
\begin{align*}
    \lim_{n\to\infty} 2^\frac{1}{n} = \lim_{n\to\infty} f(a_n) = f(0) = 2^0 = 1
\end{align*}



\section*{3.4.A}

\subsection*{Problem}
Prove the general case of $\lim_{n \to \infty} \frac{k^n}{n!}$ for a fix real number $k$.

\subsection*{Solution}
Without loss of generality, assume $k$ to be positive and let $a = \ceil{k}$ and $n \in \N : n > 2a$. Then $\frac{k^n}{n!} < \frac{a^n}{n!} = \frac{a^{2a}}{(2a)!} \times \frac{a^{n-2a}}{\frac{n!}{(2a)!}} = \frac{a^{2a}}{(2a)!} \times \underbrace{\round{\frac{a}{2a} \times \frac{a}{2a+1} \times \frac{a}{2a+2} \times \cdots}}_{n-2a \text{ terms less than } \frac{1}{2}} < \frac{a^{2a}}{(2a)!} \round{\frac{1}{2}}^{n-2a}$. Since $\frac{k^n}{n!} \geq 0$ and less than a geometric sequence with $r < 1$ it goes to 0 in the limit by the squeeze theorem. When $k < 0$, we instead approach 0 from the negative direction.



\section*{3.4.B}

\subsection*{Problem}
Prove by definition that if $a_n$ diverges to $\infty$, then if $b_n \geq a_n$, then $b_n$ also diverges to $\infty$. Similarly, if $a_n$ diverges to $-\infty$, then if $b_n \leq a_n$, then $b_n$ also diverges to $-\infty$.

\subsection*{Solution}
$\forall M \in \R : M > 0$, $\exists N \in \N$, $\forall n \in \N : n > N \implies a_n > M$, $\forall n \in \N \implies b_n > a_n$, $\forall n \in \N : n > N \implies b_n > a_n > M$. Likewise, $\forall M \in \R : M < 0$, $\exists N \in \N$, $\forall n \in \N : n < N \implies a_n < M$, $\forall n \in \N \implies b_n < a_n$, $\forall n \in \N : n < N \implies b_n < a_n < M$.



\section*{4.1.9}

\subsection*{Problem}
Restate and then negate the following definition using quantifier:

For a given sequence $a_n$, we define the sequence of partial sums $s_1$, $s_2$, $\cdots$, $s_n$ by $s_1 = a_1$, $s_n = s_{n - 1} + a_n$. We say that the series $\sum_{n = 1}^\infty a_n$ converges to $s$ and write $\sum_{n = 1}^\infty a_n = s$.

\subsection*{Solution}
\begin{align*}
    \neg \round{\exists s : (s_1 = a_1) \wedge (\forall n \in \N : n \geq 2,\ s_n = s_{n-1} + a_n) \implies \sum_{n=1}^\infty a_n = s} \\
    \nexists s : (s_1 = a_1) \wedge (\forall n \in \N : n \geq 2,\ s_n = s_{n-1} + a_n) \implies \sum_{n=1}^\infty a_n \ \text{DNE}
\end{align*}



\section*{4.1.10}

\subsection*{Problem}
Prove the following theorem:

If $\sum_{n = 1}^\infty a_n = a$ and $\sum_{n = 1}^\infty b_n = b$ then $\sum_{n = 1}^\infty (c a_n + d b_n) = ca + db$ for any $c, d \in \R$.

\subsection*{Solution}
We use the linearity property of sums to see that $\sum_{n=1}^\infty (c a_n + d b_n) = c \sum_{n=1}^\infty a_n + d \sum_{n=1}^\infty b_n = ca + db$.



\section*{4.1.12}

\subsection*{Problem}
This problem has two parts:
\begin{enumerate}
    \item Find two divergent series such that their sum is convergent.
    \item Show that if the sum and difference of two series are both convergent then so are each series individually.
\end{enumerate}

\subsection*{Solution}

\subsubsection*{1}
We have $\sum_{n=1}^\infty n = \infty$ and $\sum_{n=1}^\infty -n = -\infty$ but $\sum_{n=1}^\infty (n + (-n)) = 0$.

\subsubsection*{2}
We have $\sum_{n=1}^\infty (a_n + b_n) = c_1$ and $\sum_{n=1}^\infty (a_n - b_n) = c_2$. Then
\begin{align*}
    \sum_{n=1}^\infty (a_n + b_n) + \sum_{n=1}^\infty (a_n - b_n) &= c_1 + c_2 \\
    \sum_{n=1}^\infty (a_n + b_n + a_n - b_n) &= c_1 + c_2 \\
    \sum_{n=1}^\infty a_n &= \frac{c_1 + c_2}{2} \\
\end{align*}
So $\sum_{n=1}^\infty a_n$ converges. Likewise,
\begin{align*}
    \sum_{n=1}^\infty (a_n + b_n) - \sum_{n=1}^\infty (a_n - b_n) &= c_1 - c_2 \\
    \sum_{n=1}^\infty (a_n + b_n - a_n + b_n) &= c_1 - c_2 \\
    \sum_{n=1}^\infty b_n &= \frac{c_1 - c_2}{2} \\
\end{align*}
So $\sum_{n=1}^\infty b_n$ converges as well.



\section*{4.1.14}

\subsection*{Problem}
Show that the product of two series does not necessarily converge to the product of the limiting value of each series.

\subsection*{Solution}
By counterexample, we have $\round{\sum_{n=1}^\infty \frac{1}{n}} \round{\sum_{n=1}^\infty \frac{1}{n}} = \infty$ but $\sum_{n=1}^\infty \round{\frac{1}{n} \times \frac{1}{n}} = \sum_{n=1}^\infty n^{-2} = 2$.



\section*{4.1.15}

\subsection*{Problem}
Find the sum or show divergence of each of the following:
\begin{enumerate}
    \item $\sum_{n = 1}^\infty \frac{7}{2007^n}$
    \item $1 - \frac{1}{2} + \frac{1}{4} - \frac{1}{8} + \frac{1}{16} - \frac{1}{32} + \cdots$
    \item $\sum_{n = 1}^\infty \frac{4^{n + 1} + 5^n}{20^n}$
    \item $\sum_{k = 1}^\infty \frac{2}{4k^2 - 1}$
    \item $\sum_{n=2}^\infty \ln\round{\frac{n-1}{n}}$
\end{enumerate}

\subsection*{Solution}

\subsubsection*{1}
\begin{align*}
     & 7 \sum_{n=1}^\infty \round{\frac{1}{2007}}^n \\
    =& 7 \sum_{n=0}^\infty \round{\frac{1}{2007}}^{n+1} \\
    =& \frac{7}{2007} \sum_{n=0}^\infty \round{\frac{1}{2007}}^n \\
    =& \frac{7}{2007} \frac{1}{1-\frac{1}{2007}} \\
    =& \frac{7}{2006}
\end{align*}

\subsubsection*{2}
\begin{align*}
     & \sum_{n=0}^\infty \frac{(-1)^n}{2^n} \\
    =& \sum_{n=0}^\infty \round{-\frac{1}{2}}^n \\
    =& \frac{1}{1 - \round{-\frac{1}{2}}} \\
    =& \frac{2}{3}
\end{align*}

\subsubsection*{3}
\begin{align*}
     & \sum_{n=1}^\infty \frac{4^{n+1} + 5^n}{20^n} \\
    =& \sum_{n=1}^\infty \frac{4^{n+1}}{20^n} + \sum_{n=1}^\infty \frac{5^n}{20^n} \\
    =& \sum_{n=0}^\infty \frac{4^{n+2}}{20^{n+1}} + \sum_{n=0}^\infty \frac{5^{n+1}}{20^{n+1}} \\
    =& \frac{16}{20} \sum_{n=0}^\infty \round{\frac{4}{20}}^n + \frac{5}{20} \sum_{n=0}^\infty \round{\frac{5}{20}}^n \\
    =& \frac{4}{5} \frac{1}{1-\frac{4}{20}} + \frac{1}{4} \frac{1}{1-\frac{5}{20}} \\
    =& \frac{4}{5} \frac{20}{16} + \frac{1}{4} \frac{20}{15} \\
    =& \frac{80}{80} + \frac{20}{60} \\
    =& \frac{4}{3}
\end{align*}

\subsubsection*{4}
\begin{align*}
     & \sum_{k=1}^\infty \frac{2}{4k^2 - 1} \\
    =& \sum_{k=1}^\infty \frac{2}{(2k+1)(2k-1)}
\end{align*}

\begin{align*}
    \frac{2}{(2k+1)(2k-1)} &= \frac{a}{2k+1} + \frac{b}{2k-1} \\
    2 &= a(2k-1) + b(2k+1) \\
    2 &= 2ak - a + 2bk + b \\
    2 &= (2a+2b)k + (-a+b)
\end{align*}

\begin{align*}
    \begin{bmatrix}
        2 & 2 & 0 \\
        -1 & 1 & 2
    \end{bmatrix} \\
    \begin{bmatrix}
        1 & -1 & -2 \\
        1 & 1 & 0
    \end{bmatrix} \\
    \begin{bmatrix}
        1 & -1 & -2 \\
        0 & 2 & 2
    \end{bmatrix} \\
    \begin{bmatrix}
        1 & 0 & -1 \\
        0 & 1 & 1
    \end{bmatrix} \\
    a = -1,\ b = 1
\end{align*}

\begin{align*}
     & \sum_{k=1}^\infty \frac{1}{2k-1} - \sum_{k=1}^\infty \frac{1}{2k+1} \\
    =& 1 - \frac{1}{3} + \frac{1}{3} - \frac{1}{5} + \frac{1}{5} - \frac{1}{7} + \frac{1}{7} - \cdots \\
    =& 1
\end{align*}

\subsubsection*{5}
\begin{align*}
     & \sum_{n=2}^\infty \ln\round{\frac{n-1}{n}} \\
    =& \ln\round{\prod_{n=2}^\infty \frac{n-1}{n}} \\
    =& \ln(0)
\end{align*}
Diverges.



\section*{4.1.17}

\subsection*{Problem}
Given the sequence
\begin{align*}
    x_n = \begin{cases}
        x_1 &= 1 \\
        x_{n + 1} &= \sum_{k = 1}^n x_k
    \end{cases}
\end{align*}
\begin{enumerate}
    \item Find the general formula for $x_n$
    \item Show that $\lim_{n \to \infty} x_n = \infty$
\end{enumerate}

\subsection*{Solution}

\subsubsection*{1}
The sequence goes
\begin{align*}
    1 &= 1             \\
    1 &= 1             \\
    1+1 &= 2           \\
    1+1+2 &= 4         \\
    1+1+2+4 &= 8       \\
    1+1+2+4+8 &= 16    \\
    1+1+2+4+8+16 &= 32
\end{align*}
The general formula is $x_n = 2^{\max(0,\ n-2)}$.

\subsubsection*{2}
\begin{align*}
     & \lim_{n\to\infty} 2^{\max(0,\ n-2)} \\
    =& \frac{1}{4} \lim_{n\to\infty} 2^n \\
    =& \infty
\end{align*}



\section*{4.1.19}

\subsection*{Problem}
Given $\sum_{n = 1}^\infty a_n = a$
\begin{enumerate}
    \item Find $\sum_{n = 1}^\infty (a_n + 2a_n)$
    \item Show that $\sum_{n = 1}^\infty (a_n + 1)$ diverges
\end{enumerate}

\subsection*{Solution}

\subsubsection*{1}
\begin{align*}
     & \sum_{n=1}^\infty (a_n + 2a_n) \\
    =& 3 \sum_{n=1}^\infty a_n \\
    =& 3a
\end{align*}

\subsubsection*{2}
Because $a_n \to 0$, $a_n + 1 \to 1$ so the series diverges.



\section*{4.1.A}

\subsection*{Problem}
Write as an infinite series and then find the fraction form for the number $11.111\cdots$.

\subsection*{Solution}
\begin{align*}
     & \sum_{n=0}^\infty 10^{1-n} \\
    =& 10 \sum_{n=0}^\infty \round{\frac{1}{10}}^n \\
    =& 10 \frac{1}{1 - \frac{1}{10}} \\
    =& \frac{10}{\frac{9}{10}} \\
    =& \frac{100}{9}
\end{align*}



\section*{4.1.B}

\subsection*{Problem}
Prove that $\sum_{n=1}^\infty \frac{1}{n^2}$ is convergent by using partial sum and Squeeze Theorem. (Hint: the lower bound is zero while the upper bound is some telescoping series.)

\subsection*{Solution}



\section*{4.2.6}

\subsection*{Problem}
Assume that $\forall n \in \N$, $a_n \neq 0$ and $\sum a_n$ converges. Find the sum of each of the following:
\begin{enumerate}
    \item $\sum \frac{1}{\abs{a_n}}$
    \item $\sum \cos(a_n)$
    \item $\sum (a_n - a_{n+1})$
\end{enumerate}

\subsection*{Solution}

\subsubsection*{1}
Diverges because $a_n \to 0$ and $\frac{1}{\abs{a_n}} \to \infty$.

\subsubsection*{2}
Diverges because $a_n \to 0$ and $\cos(0) = 1$.

\subsubsection*{3}
\begin{align*}
     & \sum (a_n - a_{n+1}) \\
    =& (a_1 - a_2) + (a_2 - a_3) + (a_3 - a_4) + \cdots + (a_{n-1} - a_n) + (a_n - a_{n+1}) \\
    =& a_1 - a_{n+1}
\end{align*}
Since $a_{n+1} \to 0$, $\sum (a_n - a_{n+1}) = a_1$.



\section*{4.2.7}

\subsection*{Problem}
Given that the series $\sum a_n$ has the partial sum $s_n = \frac{n}{n+2}$ for all $n \geq 1$,
\begin{enumerate}
    \item Find $a_1$
    \item Find and expression for $a_n$ when $n \geq 2$
    \item Find $\sum_{n=1}^\infty a_n$
\end{enumerate}

\subsection*{Solution}

\subsubsection*{1}
$a_1 = \frac{1}{1+2} = \frac{1}{3}$

\subsubsection*{2}
$a_n = s_n - s_{n-1} = \frac{n}{n+2} - \frac{n-1}{(n-1)+2} = \frac{n}{n+2} - \frac{n-1}{n+1}$

\subsubsection*{3}
$\sum_{n=1}^\infty = s_n = \frac{n}{n+2}$




\end{document}
