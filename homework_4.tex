\documentclass[12pt]{article}

\usepackage{fouriernc}
\usepackage{amssymb}
\usepackage{amsmath}
\usepackage{amsfonts}
\usepackage[utf8]{inputenc}
\usepackage[T1]{fontenc}
\usepackage[margin=1in]{geometry}
\usepackage{graphicx}

\graphicspath{ {./images/} }

\newcommand{\curly}[1]{\left\{      #1 \right\}     }
\newcommand{\round}[1]{\left(       #1 \right)      }
\newcommand{\hard} [1]{\left[       #1 \right]      }
\newcommand{\abs}  [1]{\left|       #1 \right|      }
\newcommand{\floor}[1]{\left\lfloor #1 \right\rfloor}
\newcommand{\ceil} [1]{\left\lceil  #1 \right\rceil }
\newcommand{\R}    [0]{\mathbb{R}                   }
\newcommand{\Z}    [0]{\mathbb{Z}                   }
\newcommand{\N}    [0]{\mathbb{N}                   }

\setlength{\parindent}{0in}

\title{Homework 4}
\author{Tim Harding}

\begin{document}
\maketitle

\section*{3.2.8}

\subsection*{Problem}

Show that the sequence $\frac{1}{2}$, $-\frac{1}{5}$, $\frac{1}{8}$, $-\frac{1}{11}$, $\frac{1}{14}$ converges to 0.

\subsection*{Solution}



\section*{3.2.9}

\subsection*{Problem}
If $\frac{\cos(n)}{n}$ converges, prove its limit.

\subsection*{Solution}




\section*{3.2.10}

\subsection*{Problem}
Investigate the convergence of $\floor{1 - \frac{1}{n}}$.

\subsection*{Solution}




\section*{3.2.11}

\subsection*{Problem}
Show that $\forall a \in \R : \abs{a} < 1 \implies \lim_{n \to \infty} a^n = 0$. Discuss the case of $\abs{a} = 1$.

\subsection*{Solution}




\section*{3.2.12}

\subsection*{Problem}
Prove that $\lim_{n \to \infty} \frac{n^2 + n}{2n^2 + 1} = \frac{1}{2}$.

\subsection*{Solution}




\section*{3.2.15}

\subsection*{Problem}
Show that if a sequence is convergent then its limit is unique.

\subsection*{Solution}




\section*{3.2.A}

\subsection*{Problem}
Prove that $\frac{1}{n}$ is Cauchy. Specifically, when we need to get $|1/m - 1/n| < \epsilon$, try to estimate as $|1/m - 1/n| < 1/m$ and show why this would work as long as we assume $m<n$ without loss of generality (since if $n<m$, we can use $|1/m - 1/n| < 1/n$).

\subsection*{Solution}



\section*{3.3.14}

\subsection*{Problem}
Use quantifiers to negate the following definitions:

\begin{enumerate}
    \item We say that the sequence $a_n$ diverges to infinity and we write $\lim_{n \to \infty} a_n = \infty$ if for every real $B$ there exists a number $N$ such that for all $n \in \N$ with $n > N$, $a_n > B$ is true.
    \item We say that the sequence $a_n$ diverges to negative infinity and we write $\lim_{n \to \infty} a_n = -\infty$ if for every real $B$ there exists a number $N$ such that for all $n \in \N$ with $n > N$, $a_n < B$ is true.
\end{enumerate}

\subsection*{Solution}



\section*{3.3.16}

\subsection*{Problem}
Find the $n$th term of the sequence 1, 1, 2, 2, 3, 3, 4, 4, 5, 5, $\cdots$ and show that it diverges to infinity.

\subsection*{Solution}



\section*{3.3.17}

\subsection*{Problem}
Show that the sequence $a_n = \sqrt{n} - n$ diverges to negative infinity.

\subsection*{Solution}



\section*{3.3.A}

\subsection*{Problem}
Prove that $\lim_{n \to \infty} -\ln(n) = -\infty$.

\subsection*{Solution}



\section*{3.3.B}

\subsection*{Problem}
Prove that $\lim_{n \to \infty} (-1)^{n} n$ does not diverge to either $\infty$ or $-\infty$.

\subsection*{Solution}



\section*{3.4.21}

\subsection*{Problem}
Prove the following:
\begin{enumerate}
    \item $\round{\abs{a_n} \leq b_n} \wedge \round{\lim_{n \to \infty} b_n = 0} \implies \lim_{n \to \infty} a_n = 0$.
    \item $\lim_{n \to \infty} \abs{a_n} = 0 \implies \lim_{n \to \infty} a_n = 0$
\end{enumerate}

\subsection*{Solution}



\section*{3.4.22}

\subsection*{Problem}
Show that
\begin{align}
    \lim_{n \to \infty} \frac{(-1)^n \ln(n)}{\sqrt{n}} = 0
\end{align}

\subsection*{Solution}



\section*{3.4.23}

\subsection*{Problem}
Find the limit or divergence behavior of each of the following:
\begin{enumerate}
    \item $\frac{n - 2}{2n + 7}$
    \item $\frac{(-1)^n (n - 2)}{2n + 7}$
    \item $\frac{n - 2}{2n^2}$
    \item $\frac{n^2}{e^{n}}$
    \item $n^2 \round{\frac{1}{2}}^n$
    \item $\sin(\pi n + \frac{\pi}{2})$
    \item $n^\frac{1}{n}$
    \item $\round{1 - \frac{2}{n}}^n$
    \item $\sqrt{n^2 + 1} - n$
    \item $n \sin\round{\frac{1}{n}}$
\end{enumerate}

\subsection*{Solution}



\section*{3.4.25}

\subsection*{Problem}
Where possible, find the limits of
\begin{enumerate}
    \item $\round{\frac{1}{2}}^n$
    \item $\frac{27^n}{(n!)^3}$
    \item $n \round{\frac{1}{2}}^n$
    \item $\frac{n - 2}{n + 2}$
    \item $\frac{a^n}{n!}$
\end{enumerate}

\subsection*{Solution}



\section*{3.4.29}

\subsection*{Problem}
Show that if $\forall n \in \R$, $a_n \leq b_n$, $\lim_{n \to \infty} a_n = L$, $\lim_{n \to \infty} b_n = M$, then $L \leq M$. Use this result to prove that the limit is unique for a convergent sequence.

\subsection*{Solution}



\section*{3.4.31}

\subsection*{Problem}
Use the fact that if $\lim_{n \to \infty} a_n = L$ and $f : \R \to \R$ is continuous at $L$ then $\lim_{n \to \infty} f(a_n) = f(L)$ to prove that $\sqrt{1 - \frac{1}{n}}$ converges to 1.

\subsection*{Solution}



\section*{3.4.32}

\subsection*{Problem}
Use the fact that if $\lim_{n \to \infty} a_n = L$ and $f : \R \to \R$ is continuous at $L$ then $\lim_{n \to \infty} f(a_n) = f(L)$ to find the limit of $2^\frac{1}{n}$.

\subsection*{Solution}



\section*{3.4.A}

\subsection*{Problem}
Prove the general case of $\lim_{n \to \infty} \frac{k^n}{n!}$ for a fix real number $k$. (Note that when $\abs{k} \leq 1$, it is much easier, why?)

\subsection*{Solution}



\section*{3.4.B}

\subsection*{Problem}
Prove by definition that if $a_n$ diverges to $\infty$, then if $b_n \geq a_n$, then $b_n$ also diverges to $\infty$. Similarly, if $a_n$ diverges to $-\infty$, then if $b_n \leq a_n$, then $b_n$ also diverges to $-\infty$.

\subsection*{Solution}



\section*{4.1.9}

\subsection*{Problem}

\subsection*{Solution}



\section*{4.1.10}

\subsection*{Problem}

\subsection*{Solution}



\section*{4.1.12}

\subsection*{Problem}

\subsection*{Solution}



\section*{4.1.14}

\subsection*{Problem}

\subsection*{Solution}



\section*{4.1.15}

\subsection*{Problem}

\subsection*{Solution}



\section*{4.1.17}

\subsection*{Problem}

\subsection*{Solution}



\section*{4.1.19}

\subsection*{Problem}

\subsection*{Solution}



\section*{4.1.A}

\subsection*{Problem}

\subsection*{Solution}



\section*{4.1.B}

\subsection*{Problem}

\subsection*{Solution}



\section*{4.2.6}

\subsection*{Problem}

\subsection*{Solution}



\section*{4.2.7}

\subsection*{Problem}

\subsection*{Solution}




\end{document}
