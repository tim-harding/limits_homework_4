\documentclass[12pt]{article}

\usepackage{fouriernc}
\usepackage{amssymb}
\usepackage{amsmath}
\usepackage{amsfonts}
\usepackage[utf8]{inputenc}
\usepackage[T1]{fontenc}
\usepackage[margin=1in]{geometry}
\usepackage{graphicx}

\graphicspath{ {./images/} }

\newcommand{\curly}[1]{\left\{      #1 \right\}     }
\newcommand{\round}[1]{\left(       #1 \right)      }
\newcommand{\hard} [1]{\left[       #1 \right]      }
\newcommand{\abs}  [1]{\left|       #1 \right|      }
\newcommand{\floor}[1]{\left\lfloor #1 \right\rfloor}
\newcommand{\ceil} [1]{\left\lceil  #1 \right\rceil }
\newcommand{\R}    [0]{\mathbb{R}                   }
\newcommand{\Z}    [0]{\mathbb{Z}                   }
\newcommand{\N}    [0]{\mathbb{N}                   }

\setlength{\parindent}{0in}

\title{Homework 4}
\author{Tim Harding}

\begin{document}
\maketitle

\section*{3.2.8}

\subsection*{Problem}

Show that the sequence $\frac{1}{2}$, $-\frac{1}{5}$, $\frac{1}{8}$, $-\frac{1}{11}$, $\frac{1}{14}$ converges to 0.

\subsection*{Solution}
With a starting index of 0, we can write $a_n = \frac{(-1)^n}{2 + 3n}$. Then we show that $\forall \epsilon \in \R : \epsilon > 0$, $\exists N = \ceil{\frac{1}{3\epsilon} - \frac{2}{3}}$, $\forall n \in \N : n > N \implies \abs{\frac{(-1)^n}{2+3n}} = \frac{1}{2+3n} < \epsilon$.



\section*{3.2.9}

\subsection*{Problem}
If $\frac{\cos(n)}{n}$ converges, prove its limit.

\subsection*{Solution}




\section*{3.2.10}

\subsection*{Problem}
Investigate the convergence of $\floor{1 - \frac{1}{n}}$.

\subsection*{Solution}




\section*{3.2.11}

\subsection*{Problem}
Show that $\forall a \in \R : \abs{a} < 1 \implies \lim_{n \to \infty} a^n = 0$. Discuss the case of $\abs{a} = 1$.

\subsection*{Solution}




\section*{3.2.12}

\subsection*{Problem}
Prove that $\lim_{n \to \infty} \frac{n^2 + n}{2n^2 + 1} = \frac{1}{2}$.

\subsection*{Solution}




\section*{3.2.15}

\subsection*{Problem}
Show that if a sequence is convergent then its limit is unique.

\subsection*{Solution}




\section*{3.2.A}

\subsection*{Problem}
Prove that $\frac{1}{n}$ is Cauchy. Specifically, when we need to get $|1/m - 1/n| < \epsilon$, try to estimate as $|1/m - 1/n| < 1/m$ and show why this would work as long as we assume $m<n$ without loss of generality (since if $n<m$, we can use $|1/m - 1/n| < 1/n$).

\subsection*{Solution}



\section*{3.3.14}

\subsection*{Problem}
Use quantifiers to negate the following definitions:

\begin{enumerate}
    \item We say that the sequence $a_n$ diverges to infinity and we write $\lim_{n \to \infty} a_n = \infty$ if for every real $B$ there exists a number $N$ such that for all $n \in \N$ with $n > N$, $a_n > B$ is true.
    \item We say that the sequence $a_n$ diverges to negative infinity and we write $\lim_{n \to \infty} a_n = -\infty$ if for every real $B$ there exists a number $N$ such that for all $n \in \N$ with $n > N$, $a_n < B$ is true.
\end{enumerate}

\subsection*{Solution}



\section*{3.3.16}

\subsection*{Problem}
Find the $n$th term of the sequence 1, 1, 2, 2, 3, 3, 4, 4, 5, 5, $\cdots$ and show that it diverges to infinity.

\subsection*{Solution}



\section*{3.3.17}

\subsection*{Problem}
Show that the sequence $a_n = \sqrt{n} - n$ diverges to negative infinity.

\subsection*{Solution}



\section*{3.3.A}

\subsection*{Problem}
Prove that $\lim_{n \to \infty} -\ln(n) = -\infty$.

\subsection*{Solution}



\section*{3.3.B}

\subsection*{Problem}
Prove that $\lim_{n \to \infty} (-1)^{n} n$ does not diverge to either $\infty$ or $-\infty$.

\subsection*{Solution}



\section*{3.4.21}

\subsection*{Problem}
Prove the following:
\begin{enumerate}
    \item $\round{\abs{a_n} \leq b_n} \wedge \round{\lim_{n \to \infty} b_n = 0} \implies \lim_{n \to \infty} a_n = 0$.
    \item $\lim_{n \to \infty} \abs{a_n} = 0 \implies \lim_{n \to \infty} a_n = 0$
\end{enumerate}

\subsection*{Solution}



\section*{3.4.22}

\subsection*{Problem}
Show that
\begin{align}
    \lim_{n \to \infty} \frac{(-1)^n \ln(n)}{\sqrt{n}} = 0
\end{align}

\subsection*{Solution}



\section*{3.4.23}

\subsection*{Problem}
Find the limit or divergence behavior of each of the following:
\begin{enumerate}
    \item $\frac{n - 2}{2n + 7}$
    \item $\frac{(-1)^n (n - 2)}{2n + 7}$
    \item $\frac{n - 2}{2n^2}$
    \item $\frac{n^2}{e^{n}}$
    \item $n^2 \round{\frac{1}{2}}^n$
    \item $\sin(\pi n + \frac{\pi}{2})$
    \item $n^\frac{1}{n}$
    \item $\round{1 - \frac{2}{n}}^n$
    \item $\sqrt{n^2 + 1} - n$
    \item $n \sin\round{\frac{1}{n}}$
\end{enumerate}

\subsection*{Solution}



\section*{3.4.25}

\subsection*{Problem}
Where possible, find the limits of
\begin{enumerate}
    \item $\round{\frac{1}{2}}^n$
    \item $\frac{27^n}{(n!)^3}$
    \item $n \round{\frac{1}{2}}^n$
    \item $\frac{n - 2}{n + 2}$
    \item $\frac{a^n}{n!}$
\end{enumerate}

\subsection*{Solution}



\section*{3.4.29}

\subsection*{Problem}
Show that if $\forall n \in \R$, $a_n \leq b_n$, $\lim_{n \to \infty} a_n = L$, $\lim_{n \to \infty} b_n = M$, then $L \leq M$. Use this result to prove that the limit is unique for a convergent sequence.

\subsection*{Solution}



\section*{3.4.31}

\subsection*{Problem}
Use the fact that if $\lim_{n \to \infty} a_n = L$ and $f : \R \to \R$ is continuous at $L$ then $\lim_{n \to \infty} f(a_n) = f(L)$ to prove that $\sqrt{1 - \frac{1}{n}}$ converges to 1.

\subsection*{Solution}



\section*{3.4.32}

\subsection*{Problem}
Use the fact that if $\lim_{n \to \infty} a_n = L$ and $f : \R \to \R$ is continuous at $L$ then $\lim_{n \to \infty} f(a_n) = f(L)$ to find the limit of $2^\frac{1}{n}$.

\subsection*{Solution}



\section*{3.4.A}

\subsection*{Problem}
Prove the general case of $\lim_{n \to \infty} \frac{k^n}{n!}$ for a fix real number $k$. (Note that when $\abs{k} \leq 1$, it is much easier, why?)

\subsection*{Solution}



\section*{3.4.B}

\subsection*{Problem}
Prove by definition that if $a_n$ diverges to $\infty$, then if $b_n \geq a_n$, then $b_n$ also diverges to $\infty$. Similarly, if $a_n$ diverges to $-\infty$, then if $b_n \leq a_n$, then $b_n$ also diverges to $-\infty$.

\subsection*{Solution}



\section*{4.1.9}

\subsection*{Problem}
Restate and then negate the following definition using quantifier:

For a given sequence $a_n$, we define the sequence of partial sums $s_1$, $s_2$, $\cdots$, $s_n$ by $s_1 = a_1$, $s_n = s_{n - 1} + a_n$. We say that the series $\sum_{n = 1}^\infty a_n$ converges to $s$ and write $\sum_{n = 1}^\infty a_n = s$.

\subsection*{Solution}



\section*{4.1.10}

\subsection*{Problem}
Prove the following theorem:

If $\sum_{n = 1}^\infty a_n = a$ $\sum_{n = 1}^\infty b_n = b$ then $\sum_{n = 1}^\infty (c a_n + d b_n) = ca + db$ for any constants $c, d \in \R$.

\subsection*{Solution}



\section*{4.1.12}

\subsection*{Problem}
This problem has two parts:
\begin{enumerate}
    \item Find two divergent series such that their sum is convergent.
    \item Show that if the sum and difference of two series are both convergent then so are each series individually.
\end{enumerate}

\subsection*{Solution}



\section*{4.1.14}

\subsection*{Problem}
Show that the product of two series does not necessarily converge to the product of the limiting value of each series.

\subsection*{Solution}



\section*{4.1.15}

\subsection*{Problem}
Find the sum or show divergence of each of the following:
\begin{enumerate}
    \item $\sum_{n = 1}^\infty \frac{7}{2007^n}$
    \item $1 - \frac{1}{2} + \frac{1}{4} - \frac{1}{8} + \frac{1}{16} - \frac{1}{32} + \cdots$
    \item $\sum_{n = 1}^\infty \frac{4^{n + 1} + 5^n}{20^n}$
    \item $\sum_{n = 1}^\infty \frac{2}{4k^2 - 1}$
\end{enumerate}

\subsection*{Solution}



\section*{4.1.17}

\subsection*{Problem}
Given the sequence
\begin{align}
    x_n = \begin{cases}
        x_1 &= 1 \\
        x_{n + 1} &= \sum_{k = 1}^n x_k
    \end{cases}
\end{align}
\begin{enumerate}
    \item Find the general formula for $x_n$
    \item Show that $\lim_{n \to \infty} x_n = \infty$
\end{enumerate}

\subsection*{Solution}



\section*{4.1.19}

\subsection*{Problem}
Given $\sum_{n = 1}^\infty a_n = a$
\begin{enumerate}
    \item Find $\sum_{n = 1}^\infty (a_n + 2a_n)$
    \item Show that $\sum_{n = 1}^\infty (a_n + 1)$ diverges
\end{enumerate}

\subsection*{Solution}



\section*{4.1.A}

\subsection*{Problem}
Write as an infinite series and then find the fraction form for the number $11.111\cdots$.

\subsection*{Solution}



\section*{4.1.B}

\subsection*{Problem}
Prove that $\sum_{n=1}^\infty \frac{1}{n^2}$ is convergent by using partial sum and Squeeze Theorem. (Hint: the lower bound is zero while the upper bound is some telescoping series.)

\subsection*{Solution}



\section*{4.2.6}

\subsection*{Problem}
Assume that $\forall n \in \N$, $a_n \neq 0$ and $\sum a_n$ converges. Find the sum of each of the following:
\begin{enumerate}
    \item $\sum \frac{1}{\abs{a_n}}$
    \item $\sum \cos(a_n)$
    \item $\sum (a_n - a_{n+1})$
\end{enumerate}

\subsection*{Solution}



\section*{4.2.7}

\subsection*{Problem}
Given that the series $\sum a_n$ has the partial sum $s_n = \frac{n}{n+2}$ for all $n \geq 1$,
\begin{enumerate}
    \item Find $a_1$
    \item Find and expression for $a_n$ when $n \geq 2$
    \item Find $\sum_{n=1}^\infty a_n$
\end{enumerate}

\subsection*{Solution}




\end{document}
